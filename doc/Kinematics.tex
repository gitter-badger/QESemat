\documentclass[a4paper,12pt,landscape]{article}
\usepackage[utf8]{inputenc}
\usepackage[english,russian]{babel}
\usepackage{textcomp}
\usepackage{amssymb,amsmath}
\usepackage{feynmf}
\usepackage{cite}
\usepackage[dvips,final]{graphicx}
\usepackage{xcolor}
\usepackage{geometry}
	\geometry{left=1cm}
	\geometry{right=1cm}
	\geometry{top=1cm}
	\geometry{bottom=1cm}
\usepackage{indentfirst}
\author{Ольга~Николаевна~Петрова}
\title{Кинематика}
\date{2013}
\parindent=1.27cm
\abovecaptionskip=0mm
\belowcaptionskip=0mm
\definecolor{MYblue}{rgb}{0.0470,0,0.5294}
\definecolor{MYgreen}{rgb}{0.0470,0.5294,0}
\definecolor{MYyellow}{rgb}{0.7,0.6,0}
\makeatletter
\renewcommand{\l@section}{\@dottedtocline{1}{1.5em}{2.3em}}
\renewcommand{\tableofcontents}{{\centering\section*{\contentsname}}\@starttoc{toc}}
\renewcommand{\listoffigures}{{\centering\section*{\listfigurename}}\@starttoc{lof}}
%\renewcommand{\listoftables}{{\centering\section*{\listtablename}}\@starttoc{lot}}
\renewcommand{\@biblabel}[1]{#1.}
\renewcommand{\labelenumii}{\theenumii)}
\renewcommand{\theenumii}{\arabic{enumii}}
%\newcommand{\ralph}[1]{\@ralph{\@nameuse{c@#1}}}
%\newcommand{\@ralph}[1]{%
%	\ifcase #1\or %
%		а\or б\or в\or г\or д\or е\or ж\or з\or и\or к\or л\or м\or н\or о\or п\or р\or с\or т\or у\or ф\or х\or ц\or ч\or ш\or щ\or э\or ю\or я%
%	\else\@ctrerr %
%	\fi}
%\renewcommand{\theenumii}{\ralph{enumii}}
\newcommand{\@makefigurecaption}[2]{%
	\vspace{\abovecaptionskip}
	\sbox{\@tempboxa}{#1.~#2}
	\ifdim \wd\@tempboxa>\hsize
   		#1.~#2\par
	\else
   		\global\@minipagefalse
   		\hbox to \hsize {\hfil#1.~#2\hfil}%
	\fi
	\vspace{\belowcaptionskip}}
\renewcommand{\figure}{\let\@makecaption\@makefigurecaption\@float{figure}}
\newcommand{\@maketablecaption}[2]{%
	\vspace{\abovecaptionskip}
	\sbox{\@tempboxa}{#2}
	\hfill #1\par
	\ifdim \wd\@tempboxa>\hsize
   		#2\par
	\else
   		\global\@minipagefalse
   		\hbox to \hsize {\hfil#2\hfil}%
	\fi
	\vspace{\belowcaptionskip}}
\renewcommand{\table}{\let\@makecaption\@maketablecaption\@float{table}}
\makeatother
%\numberwithin{equation}{section}
\renewcommand{\baselinestretch}{1.5}
\frenchspacing
\setlength{\unitlength}{1mm}
\begin{document}
\pagestyle{empty}
\subsection*{Квазиупругое}
\subsubsection*{рассеяние на неподвижном нуклоне}
Лабораторная система отсчёта:
\begin{figure}[!ht]
\begin{center}
\begin{fmffile}{FD1}
\begin{fmfchar*}(40,40)
\fmfleftn{i}{2} 
\fmfrightn{o}{2}
\fmflabel{$N_{i}({\color{blue}p_{i}})$}{i1}
\fmflabel{$\nu_{l}({\color{blue}p_{\nu}})$}{i2}
\fmflabel{$N_{f}({\color{blue}p_{f}})$}{o1}
\fmflabel{$l({\color{blue}p_{l}})$}{o2}
\fmf{fermion}{i1,mi,o1}
\fmf{boson,lab.side=left,label=$W$}{mi,mo}
\fmf{boson,lab.side=right,label=$\color{blue}q$}{mi,mo}
\fmf{fermion}{i2,mo,o2}
\fmfdot{mi,mo}
\end{fmfchar*}
\end{fmffile}
\caption{$\nu_{l}({\color{blue}p_{\nu}=(w,\vec{p}_{\nu})})+N_{i}({\color{blue}p_{i}=(M,\vec{0})})\to{}l({\color{blue}p_{l}=(E,\vec{p})})+N_{f}({\color{blue}p_{f}=(E_{f},\vec{p}_{f})})$, где $|\vec{p}_{\nu}|=w$}
\label{FD1}
\end{center}
\vspace{-0.5cm}
\end{figure}

Закон сохранения 4-импульса:
\begin{align*}
p_{\nu}+p_{i}=p_{l}+p_{f}\Rightarrow
\left\lbrace
\begin{aligned}
w+M={}&E+E_{f}\\
\vec{p}_{\nu}={}&\vec{p}+\vec{p}_{f}
\end{aligned}\right.\Rightarrow
\end{align*}
%
\begin{align*}
\left\lbrace
\begin{aligned}
w+M-E={}&E_{f}\\
\vec{p}_{f}={}&\vec{p}_{\nu}-\vec{p}
\end{aligned}\right.\Rightarrow
\left\lbrace
\begin{aligned}
&{\color{MYgreen}w^{2}+M^{2}+m^{2}+p^{2}+2wM-2wE-2ME={}}w^{2}+M^{2}+E^{2}+2wM-2wE-2ME=E_{f}^{2}{\color{MYgreen}{}=M_{f}^{2}+p_{f}^{2}}\\
&p_{f}^{2}=p_{\nu}^{2}+p^{2}-2\vec{p}_{\nu}\vec{p}{\color{MYgreen}{}=p_{\nu}^{2}+p^{2}-2|\vec{p}_{\nu}|pz=w^{2}+p^{2}-2wpz},\text{ где }{\color{brown}z=\cos{\theta}}\Rightarrow \boxed{z\in[-1;1]}.
\end{aligned}\right.
\end{align*}
%
\begin{align*}
w^{2}+M^{2}+m^{2}+p^{2}+2wM-2wE-2ME={}&M_{f}^{2}+w^{2}+p^{2}-2wpz\\
M^{2}+m^{2}+2wM-2wE-2ME={}&M_{f}^{2}-2wpz{\color{MYgreen}{}=M^{2}+\Delta{M}^{2}-2wpz},\text{ где }{\color{brown}\Delta{M}^{2}=M_{f}^{2}-M^{2}}\\
m^{2}-\Delta{M}^{2}+2wM-2wE+2wpz-2ME={}&0\\
\tilde{m}^{2}+2w(M-E+pz)={}&2ME,\text{ где }{\color{brown}\tilde{m}^{2}=m^{2}-\Delta{M}^{2}}\\
w(M-E+pz)={}&ME-\tilde{m}^{2}/2\\
w={}&\boxed{\frac{ME-\tilde{m}^{2}/2}{M-E+pz}}.
\end{align*}

$w$ --- энергия нейтрино $\Rightarrow$ $w\ge{}0$.
\begin{align}
\left\lbrace
\begin{aligned}
&\left[
\begin{aligned}
\left\lbrace
\begin{aligned}
&ME-\tilde{m}^{2}/2\ge{}0\\
&M-E+pz>0
\end{aligned}\right.\\
\left\lbrace
\begin{aligned}
&ME-\tilde{m}^{2}/2<0\\
&M-E+pz<0
\end{aligned}\right.
\end{aligned}\right.\\
&z\in[-1;1]
\end{aligned}\right.\Rightarrow
\left\lbrace
\begin{aligned}
&\left[
\begin{aligned}
&\left\lbrace
\begin{aligned}
&E\ge{}\frac{\tilde{m}^{2}}{2M}\\
&pz+M>E
\end{aligned}\right.\\
&\left\lbrace
\begin{aligned}
&E<\frac{\tilde{m}^{2}}{2M}\\
&pz+M<E
\end{aligned}\right.
\end{aligned}\right.\\
&z\in[-1;1].
\end{aligned}\right.
\label{all}
\end{align}

Будем считать, что $M_{f}=M\Rightarrow \Delta{M}^{2}=0\Rightarrow \tilde{m}^{2}=m^{2}$. Тогда части первого неравенства положительны.
\begin{align*}
{\color{MYgreen}m^{2}+p^{2}={}}E^{2}\ge{}\frac{m^{4}}{(2M)^{2}}\Rightarrow
p^{2}\ge{}\frac{m^{4}}{(2M)^{2}}-m^{2}{\color{MYgreen}{}=m^{2}\left(\left(\frac{m}{2M}\right)^{2}-1\right)}.
\end{align*}

$m<2M$ даже для тау-лептона $\Rightarrow$ $\left(\dfrac{m}{2M}\right)^{2}-1<0$. $p^{2}\ge{}0$, так что первое неравенство выполняется всегда, а третье - никогда:
\begin{align*}
\left[
\begin{aligned}
&\left\lbrace
\begin{aligned}
&E\ge{}\frac{m^{2}}{2M}\\
&pz+M>E
\end{aligned}\right.\\
&\left\lbrace
\begin{aligned}
&E<\frac{m^{2}}{2M}\\
&pz+M<E
\end{aligned}\right.
\end{aligned}\right.\Rightarrow
\left[
\begin{aligned}
&\left\lbrace
\begin{aligned}
&p\in(-\infty;+\infty)\\
&pz+M>E
\end{aligned}\right.\\
&\left\lbrace
\begin{aligned}
&p\in\varnothing\\
&pz+M<E
\end{aligned}\right.
\end{aligned}\right.\Rightarrow
\boxed{pz+M>E}
\end{align*}
\begin{align}
pz+M>E{\color{MYgreen}{}\ge{}0}\Rightarrow
\left\lbrace
\begin{aligned}
&pz+M>0\\
&p^{2}z^{2}+M^{2}+2Mpz>E^{2}
\end{aligned}\right.\Rightarrow
\left\lbrace
\begin{aligned}
&z>-M/p\\
&z\in(-\infty;z_{-})\cup(z_{+};+\infty).
\end{aligned}\right.
\end{align}

Корни второго неравенства:
\begin{align*}
p^{2}z^{2}+M^{2}+2Mpz=E^{2}\Rightarrow{}& p^{2}z^{2}+2Mpz+M^{2}-E^{2}=0\\
z_{\pm}=\frac{-Mp\pm\sqrt{M^{2}p^{2}-p^{2}(M^{2}-E^{2})}}{p^{2}}={}&\frac{-M\pm\sqrt{M^{2}-M^{2}+E^{2}}}{p}=\boxed{\frac{-M\pm{}E}{p}.}
\end{align*}

(\ref{all}) сводится к:
\begin{align*}
\left\lbrace
\begin{aligned}
&z\in\left(-\frac{M}{p};+\infty\right)\\
&z\in\left(-\infty;\frac{-M-E}{p}\right)\cup\left(\frac{-M+E}{p};+\infty\right)\\
&z\in[-1;1]
\end{aligned}\right.\Rightarrow
\left\lbrace
\begin{aligned}
&z\in\left(\frac{-M+E}{p};+\infty\right)\\
&z\in[-1;1]
\end{aligned}\right.\Rightarrow
\left\lbrace
\begin{aligned}
&z\in(z_{+};+\infty)\\
&z\in[-1;1].
\end{aligned}\right.
\end{align*}
%
\begin{align}
\boxed{z\in
\begin{cases}
[-1;1],&\text{ если }z_{+}<-1;\\
[z_{+};1],&\text{ если }z_{+}\in[-1;1];\\
\varnothing,&\text{ если }z_{+}>1.
\end{cases}}
\label{zcases}
\end{align}
%
\begin{enumerate}
\item
\begin{align*}
{\color{MYgreen}\frac{-M+\sqrt{m^{2}+p^{2}}}{p}=\frac{-M+E}{p}={}}z_{+}<-1\Rightarrow
-M+\sqrt{m^{2}+p^{2}}<-p\Rightarrow
M-p>\sqrt{m^{2}+p^{2}}
\end{align*}
\begin{align*}
M-p>\sqrt{m^{2}+p^{2}}{\color{MYgreen}{}>0}\Rightarrow
\left\lbrace
\begin{aligned}
&M-p>0\\
&M^{2}+p^{2}-2Mp>m^{2}+p^{2}
\end{aligned}\right.\Rightarrow
\left\lbrace
\begin{aligned}
&p<M\\
&M^{2}-m^{2}>2Mp
\end{aligned}\right.\Rightarrow
\left\lbrace
\begin{aligned}
&p<M\\
&p<\frac{M^{2}-m^{2}}{2M}{\color{MYgreen}{}=\frac{M}{2}-\frac{m^{2}}{2M}<M}
\end{aligned}\right.\Rightarrow
p<\zeta,\text{ где }{\color{brown}\zeta=\dfrac{M^{2}-m^{2}}{2M}}.
\end{align*}
\item[3.]
\begin{align*}
{\color{MYgreen}\frac{-M+\sqrt{m^{2}+p^{2}}}{p}={}}z_{+}>1&{}\Rightarrow
-M+\sqrt{m^{2}+p^{2}}>p\Rightarrow
\sqrt{m^{2}+p^{2}}>p+M\\
m^{2}+p^{2}>p^{2}+M^{2}+2Mp&{}\Rightarrow
m^{2}-M^{2}>2Mp\Rightarrow
p<\frac{m^{2}-M^{2}}{2M}{\color{MYgreen}{}=-\zeta}.
\end{align*}
\item Пересечение областей, дополняющих 1 и 3:
\begin{align*}
\left\lbrace
\begin{aligned}
&p\ge{}\zeta\\
&p\ge{}-\zeta
\end{aligned}\right.\Rightarrow
p\ge{}|\zeta|.
\end{align*}
\end{enumerate}

Тогда (\ref{zcases}) будет выглядеть так:
\begin{align}
\boxed{z\in
\begin{cases}
[-1;1],&\text{ если }p<\zeta;\\
[z_{+};1],&\text{ если }p\ge{}|\zeta|;\\
\varnothing,&\text{ если }p<-\zeta.
\end{cases}}
\label{ini}
\end{align}

\begin{align}
z\in
\begin{cases}
\begin{cases}
[-1;1],&\text{ если }p<|\zeta|;\\
[z_{+};1],&\text{ если }p\ge{}|\zeta|;\\
\varnothing,&\text{ если }p<-|\zeta|,
\end{cases}&\text{ если }\zeta>0;\\
\begin{cases}
[-1;1],&\text{ если }p<-|\zeta|;\\
[z_{+};1],&\text{ если }p\ge{}|\zeta|;\\
\varnothing,&\text{ если }p<|\zeta|,
\end{cases}&\text{ если }\zeta<0\\
\end{cases}\Rightarrow
z\in
\begin{cases}
\begin{cases}
[-1;1],&\text{ если }p<|\zeta|;\\
[z_{+};1],&\text{ если }p\ge{}|\zeta|,
\end{cases}&\text{ если }\zeta>0;\\
\begin{cases}
[z_{+};1],&\text{ если }p\ge{}|\zeta|;\\
\varnothing,&\text{ если }p<|\zeta|,
\end{cases}&\text{ если }\zeta<0\\
\end{cases}\Rightarrow
z\in
\begin{cases}
\begin{cases}
[-1;1],&\text{ если }\zeta>0;\\
\varnothing,&\text{ если }\zeta<0,
\end{cases}&\text{ если }p<|\zeta|;\\
[z_{+};1],&\text{ если }p\ge{}|\zeta|.
\end{cases}
\label{fin}
\end{align}
\begin{align*}
\boxed{w=\dfrac{ME-m^{2}/2}{M-E+pz}}
\end{align*}
\begin{align*}
w_{\min}=\frac{ME-m^{2}/2}{M-E+pz_{\max}}\Rightarrow
w_{\min}=\frac{ME-m^{2}/2}{M-E+p},\text{ при }p<|\zeta|,\text{ если }\zeta>0\text{ или }p\ge{}|\zeta|.
\end{align*}
\begin{align*}
w_{\max}=\frac{ME-m^{2}/2}{M-E+pz_{\min}}\Rightarrow
w_{\max}=
\begin{cases}
\dfrac{ME-m^{2}/2}{M-E-p},&\text{ если }p<|\zeta|,\zeta>0;\\
\dfrac{ME-m^{2}/2}{M-E+pz_{+}}{\color{MYgreen}{}=\dfrac{ME-m^{2}/2}{M-E+p\frac{-M+E}{p}}=\dfrac{ME-m^{2}/2}{0}=+\infty},&\text{ если }p\ge{}|\zeta|.
\end{cases}
\end{align*}
\begin{align*}
\boxed{w\in
\begin{cases}
\left[\dfrac{ME-m^{2}/2}{M-E+p};\dfrac{ME-m^{2}/2}{M-E-p}\right],&\text{ если }p<|\zeta|,\zeta>0;\\
\left[\dfrac{ME-m^{2}/2}{M-E+p};+\infty\right),&\text{ если }p\ge{}|\zeta|.
\end{cases}}
\end{align*}

$\zeta<0$ только для тау-лептона.
\end{document}
